\input{../preambule.tex}

\title{Modal logic, $\mu$-calcul, parity games and others}
\author{Diego Dorn}

\begin{document}

\maketitle

\section{Beyond first order logic}

The goal of this section is to introduce the reader
to some extensions of first order logic (FO),
and most importantly to develop monadic second order logic.

\subsection{Second order logic}

In first order logic, quantifications happen only over the
domain elements.
Second order logic (SO) is a generalization of FO,
where quantification over relations are allowed.
It adds second order variables, usually denoted by capital
letters, that are interpreted by relations, that is,
if the relation is of arity $n$, subsets of $\Mm^n$.
\begin{itemize}
    \item $t = t'$ when $t, t'$ are terms.
    \item $R(t_1, \dots, t_n)$ when $t_1, \dots, t_n$ are terms
        and $R \in \sigma$ is a relation of arity $n$.
    \item $X(t_1, \dots, t_n)$ when $t_1, \dots, t_n$ are terms
        and $X$ is a second-order variable of arity $n$.
\end{itemize}
For instance, the following formula are syntaxically valid:
\begin{itemize}
    \item $\forall P~ \exists x~ \forall y~ P(x) \implies P(y)$,
        where $P$ is a variable of arity $1$.
    \item $\forall A ~ \exists x ~
            (A(x) \implies \forall y ~
                (A(y) \implies ~ x \leq y) $,
        assuming that $\leq$ is part of the language
        and $A$ is a variable of arity $1$.
        This expresses that every subset has a minimal element.
    \item $\exists E\, \forall x\, \forall y \,
        (E(x, y) \implies R(x, y)) \wedge \forall x \, \exists! y \, E(x, y)$
        where $E$ is a relation of arity $2$ and $R$ is a relation of the language.
        This express the fact that if $R$ is considered as the edges of a graph,
        then there is a perfect matching.
\end{itemize}

\begin{definition}
    Given a vocabulary $\sigma$ consisting of relations
    and constants symbols,
    the \emph{terms} of SO are constants symbols and first order variables.

    The \emph{atomic formulas} are of the form:
    \begin{itemize}
        \item $t = t'$ when $t, t'$ are terms.
        \item $R(t_1, \dots, t_n)$ when $t_1, \dots, t_n$ are terms
            and $R \in \sigma$ is a relation of arity $n$.
        \item $X(t_1, \dots, t_n)$ when $t_1, \dots, t_n$ are terms
            and $X$ is a second-order variable of arity $n$.
    \end{itemize}

    The set of SO \emph{formulas} is the smallest set that contains
    all atomic formulas and is closed under:
    \begin{itemize}
        \item boolean operators: $\neg$, $\wedge$, $\vee$ and first order quantification
        \item second order quantification: if $\phi(\vec{x}, Y, \vec{X})$
            is a SO formula, then $\forall Y \phi(\vec{x}, Y, \vec{X})$
            and $\exists Y \phi(\vec{x}, Y, \vec{X})$ are SO formulas.
    \end{itemize}

    The \emph{semantic} of SO logic is defined similarly to FO logic
    so we only need to define the semantic for new constructs.
    Let $\Mm$ be a $\s$-structure and $\phi(x_1, \dots, x_k, X_1, \dots, X_n)$
    a SO formula. We define $\Mm \models \phi(\vec{b}, \vec{B})$
    where $b \in \Mm^k$ is a tuple of elements of $\Mm$
    and if $X_i$ is of arity $n_i$, then $B_i$ is a subset of $\Mm^{n_i}$.

    \begin{itemize}
        \item If $\phi(x_1, \dots, x_k, X)$ is $X(t_1, \dots, t_n)$
            with $X$ a second-order variable of arity $n$
            and $t_1, \dots, t_n$ terms with free variables
            among $x_1, \dots, x_i$, then
            $\Mm \models \phi(\vec{b}, B)$
            if $(t_1^\Mm\vec{b}), \dots, t_k^\Mm(\vec{b}))$ is in $B$.
        \item If $\phi(\vec{x}, Y, \vec{X})$ is $\forall Y \psi(\vec{x}, Y, \vec{X})$
            with $Y$ a second-order variable of arity $n$
            then $\Mm \models \phi(\vec{b}, B)$
            if for all $C \subseteq \Mm^n$, $\Mm \models \psi(\vec{x}, C, \vec{X})$
        \item If $\phi(\vec{x}, Y, \vec{X})$ is $\exists Y \psi(\vec{x}, Y, \vec{X})$
            with $Y$ a second-order variable of arity $n$
            then $\Mm \models \phi(\vec{b}, B)$
            if for some $C \subseteq \Mm^n$, $\Mm \models \psi(\vec{x}, C, \vec{X})$
    \end{itemize}
\end{definition}

We will not study much of second order logic, but instead
look at one of its fragment, monadic second order logic.

\subsection{Monadic second order logic}

Monadic second order logic, or MSO is an extension of
fisrt order logic and a restriction of second order logic.
In MSO, valid formulas are formulas of second order logic
where second order quantification happens only on unary relations.
This corresponds to be able to quantify over elements
of the domain (first-order quantification)
or over subsets of the domain (second-order quantification).

The semantics of MSO are the same as SO semantics.

There are two things that are remarkable about
this logic, when considered together:
\begin{itemize}
    \item MSO is very expressive. It can for instance
        describe problems at every level of the polynomial hierarchy ($\PHierachy$).
    \item MSO is decidable for a large class of models, for instance
        infinte strings and trees.
\end{itemize}


\section{Decidability of MSO on strings}






































\iffalse
\section{Introduction}

\subsection{The parity game}
\begin{definition}
    The \emph{parity game} is an inifinite two player game
    played on a directed graph $G = (V, E)$. Let $V_0, V_1$
    be a partition of the vertices of $G$, and $p: V \to \N$
    any function, called the \emph{priority function}.
    Let also $v_0 \in V$ be the strating point of the game.

    The game goes as follows:
    \begin{itemize}
        \item It starts at $v_0$
        \item At a given turn $i$, if $v_i \in V_0$,
            player 0 decides to move to an adjacent node
            $v_{i+1}$. Otherwise if $v_i \in V_1$, it is
            player 1 that picks a neighbourg.
        \item After $\w$ turns, they produce a sequence of vertices
            $(v_n)_{n \in \N} \in V$. Let $V^\infty$ be the set of
            vertices that have been visited infinitely many times.
            If $\max \setst{p(v)}{v \in V^\infty}$ is even, player 0
            wins. If it is odd, player 1 wins.
    \end{itemize}
\end{definition}

\[
    W = \setst{b \in T}{
        \exists z \in V ~ \forall z' \in V \left(
            \begin{array}{c}
                p(z) \even
                \\ \wedge \\
                \existsinf i,~ v_i = z
                \\ \wedge \\
                p(z') > p(z) \implies \exists N \forall n > N v_n \neq z'
            \end{array}
        \right)
    }
\]
\fi

\newpage
\nocite{*}
\bibliographystyle{apalike}
\bibliography{bibliography.bib}


\end{document}